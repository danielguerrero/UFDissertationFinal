\chapter{Conclusions} \label{chapter:conclusions}
%Motivation
The Higgs energy potential is connected to the physics of the very early universe, the mechanism electroweak symmetry breaking, and the stability of the Universe. The measurement of Higgs boson self-coupling $\mathrm{\lambda_{HHH}}$ is crucial for the reconstruction of this potential. If we observe any deviations of $\mathrm{\lambda_{HHH}}$ relative to the SM, this could have profound consequences in our understanding of nature. At the LHC, non-resonant HH production gives direct access to this coupling. However, this process has a small SM cross section $\mathrm{\sigma_{SM}^{HH}}$, making it a very rare to see at the current data luminosities. The study of the HH process can shed light on the nature of the scalar sector. By being sensitive to BSM physics, we use the HH signature as a probe to study it and potentially discover it already with the current datasets.

%The analysis 2016 vs Run-2 new analysis strategy
During the LHC Run-2 period, the CMS experiment collected an unprecedented amount of pp collision data at $\mathrm{\sqrt{s}= 13}$ TeV totaling 138 $\mathrm{fb^{-1}}$. This dissertation presents an original Run-2 search for non-resonant HH production in the $\bbbb$ channel, exploring both the gluon fusion and vector boson fusion production mechanisms. It implements innovative techniques in the areas of object identification, event categorization, signal identification, and background modeling. In particular, it deploys, for the first time in CMS, a full data-driven background modeling using an ML algorithm known as BDT-reweighting. The presented result improves the expected analysis sensitivity by a factor of 5 with respect to previous 2016-only result, where a factor of 2 comes from the increase of integrated luminosity and a factor 3 comes from the better and new analysis methods. The observed data does not show an excess of events with respect to the background-only hypothesis, and 95\% CL upper limits on the SM and BSM production cross sections are set. These results are complemented with measurements of the couplings associated to the HH process using one-dimensional and two-dimensional likelihood scans.

%The result and what implies for Run-2 HH
The work of this dissertation sets an observed limit on the SM HH production cross section at $3.6$ x $\mathrm{\sigma_{SM}^{HH}}$. At the time of writing this thesis, it is the best single-channel limit at the LHC. It sets tight constraints on the production cross section of anomalous Higgs couplings and benchmark signals for new physics in an EFT scenario. It excludes the values of the Higgs boson self-coupling modifier outside the range $-2.3<\kl<9.4$ at 95\% CL, and the values of the coupling modifier of the di-vector-boson-di-Higgs-boson interaction outside the range $-0.1<\kvv<2.2$ at 95\% CL. Furthermore, the measurement of the couplings show the observation compatible with the SM hypothesis at 95\% CL. 

The results obtained in this dissertation will be a crucial piece of the Run-2 combination of all non-resonant searches planned at the end of 2021 / early 2022. Just by statistically combining the results of the three most sensitive channels ($\mathrm{b\overline{b}b\overline{b}}$, $\mathrm{b\overline{b}\gamma\gamma}$ and $\mathrm{b\overline{b}\tau^{+}\tau^{-}}$), the CMS experiment is expected to set a constraint on the SM cross section at 2--3 x $\mathrm{\sigma_{SM}^{HH}}$, exceeding preliminary expectations from the extrapolation of the 2016 HH combination ($\sim13$ x $\mathrm{\sigma_{SM}^{HH}}$) to the Run-2 integrated luminosity ($\sim7$ x $\mathrm{\sigma_{SM}^{HH}}$). Therefore, there is almost a factor of 3 sensitivity gain (on top of the luminosity), which has been achieved through new and sophisticated analysis methods in every channel.

%The Run-3 
The LHC Run-3 is expected to start in early 2022. During this 3-4 year running period, the LHC will produce pp collisions at $\mathrm{\sqrt{s}=}$13 or 14 TeV, and will collect a dataset which will at least double the integrated luminosity of the one collected in Run-2. It will be a great opportunity to establish a solid HH strategy with multiple decay channels and production modes. It will be essential to develop new strategies to maximize the sensitivity of the searches for SM and BSM physics with this unprecedented dataset. Based on the experience of this work, it will be important to address the current limitations of the $\mathrm{b\overline{b}b\overline{b}}$ channel: acceptance (the trigger, b jet identification), systematic uncertainties (accuracy of the background model and control region statistical precision), and multijet background rejection.

The Run 3 era will be the playground for new ideas and not just more refined techniques. There is an ongoing effort with the goal of developing a very efficient $\mathrm{HH\rightarrow b\overline{b}b\overline{b}}$ Run-3 trigger. The new trigger requires only two jets to be identified as b jets instead of the three required in the Run-2 triggers, without increasing the jet $\pt$ requirements to control the trigger rate. The `2b' requirement will allow for additional `b-tag' control region data to be collected, to derive and validate the future data-driven background model. The new trigger is also being developed using the novel jet flavor tagging algorithm ParticleNet~\cite{Qu:2019gqs}, which uses a state-of-the-art machine learning architecture based on point clouds. This tagger has demonstrated to surpass the performance of the standard Run-2 algorithms. The same tagger could be used for the offline analysis. Furthermore, the analysis background model based on BDT-reweighting (or other similar methods) is limited systematically by the statistical power of the control and validation region data. The lack of statistical power can be addressed by using the information of the extra `2b' region. An alternative way for exploration is to use data-driven techniques that could generate a synthetic dataset from real data at high fidelity. These techniques can be ML generative models such as generative adversarial networks (GANs), which are currently studied to speed up the huge production of MC simulation needs for the HL-LHC. Lastly, a large dataset of data-driven background model events will allow for more sophisticated discriminants (e.g. DNNs) to be trained, and used for background rejection and signal extraction. 

%and beyond
The HL-LHC operation is expected to start in 2027 and continue for at least a decade, collecting 14 TeV pp collision data at higher luminosities. Consequently, this will be the ideal opportunity to measure SM HH production given the large expected dataset ($\sim$3000-4000~$\mathrm{fb^{-1}}$). According to the last HL-LHC projection at 3000~$\mathrm{fb^{-1}}$, the expected significance for this process is 3.0 and 2.6 standard deviations for ATLAS and CMS, respectively. The ATLAS and CMS combined result yields an expected significance of 4.0 standard deviations, thus the evidence for HH production will be at reach. The actual results will be surely better than these projections with the progress and innovation of future experimental methods. Based on the improvements already seen in Run-2, a larger dataset will allow for more sensitive results to be obtained. However, the experimental challenges at high PU will need to be overcome. 

%Last paragraph
The work of this dissertation shows that the $\bbbb$ channel can be one of the most sensitive HH channels at the LHC, despite being considered very challenging a few years ago. It opens the door to promising prospects for a solid and long-term program starting from Run-2 combination to the HL-LHC era, with the eventual possibility to deepen our understanding of the fundamental laws of nature. My wish is for this work to serve as a stepping stone and a guide for the future searches in this channel.